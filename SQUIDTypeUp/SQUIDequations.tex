\documentclass[12pt]{article}

\author{Cliff Sun}

\title{General Overview Behind Non-Ideal SQUID Python Program}

\begin{document}

\maketitle

\begin{center} The governing equation \end{center}

\begin{equation}
    I = I_{c} \sin \Delta \varphi
\end{equation}

\begin{center} The evolution of the phase difference \end{center}

\begin{equation}
\Delta \varphi = \Delta\varphi_{0} + 2 \pi B*x
\end{equation}

\begin{center} Such that B is the applied Magnetic Field, \(\Delta\varphi_{0}\) is the initial phase difference, 
and x is the spatial distance relative to either of the ends of the SQUID \end{center}

\begin{center} The approximation formula I used for the current of each Josephson Junction \end{center}
\begin{equation}
\sum a_{n}dx    
\end{equation}

\begin{equation}
current = current + J_d \sin (\Delta\varphi_0 + 2 \pi B*x)dx
\end{equation}

\begin{center}
    \(J_d\) is the current density at that Junction \\
    \(\Delta\varphi_0\) = initial phase difference at that Junction \\
    B = normalized magentic flux quanta \\
    location = spatial location of the junction with respect to the ends of the SQUID
\end{center}

\begin{equation}
B = (2\lambda + d) \phi_d / \phi_0
\end{equation}

\begin{center} With units \((unit\:length)^{-1}\) \end{center}

\end{document}  